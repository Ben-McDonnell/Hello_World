\documentclass[12pt,amstex,amssymb,amscd]{amsart}
\usepackage[latin1]{inputenc}
\usepackage{amsmath}
\usepackage{amsthm}
\usepackage{amsfonts}
\usepackage{amssymb}
\usepackage{graphicx}
\usepackage{tikz-cd}

\newtheorem{theorem}{Theorem}
\newtheorem{proposition}{Proposition}
\newtheorem{corollary}{Corollary}
\newtheorem{definition}{Definition}
\newtheorem{lemma}{Lemma}
\theoremstyle{remark}
\newtheorem*{remark}{Remark}
\DeclareMathOperator{\Hom}{Hom}
\DeclareMathOperator{\End}{End}
\DeclareMathOperator{\Mod}{-Mod}
\newcommand{\mc}{\mathcal}
\newcommand{\mf}{\mathfrak}
\newcommand{\bb}{\mathbb}

\begin{document}
	\title{Uniqueness of Soergel's Grading on Category $\mc{O}$}
	\author{Ben McDonnell}
	\date{}
	\maketitle
	
	\section{Introduction}
	This paper is about graded covers of Category $\mc{O}$ and the deformed version of $\mc{O}$. The concept of a graded cover is similar to a grading on a category of modules, and the correct definition is given in [RS14]. The deformed version of Category $\mc{O}$ was originally defined in [Soe90] and in some sense consists of the projective modules in $\mc{O}$. In his 1990 paper, Soergel constructed a grading on the usual $\mc{O}$ which is compatible with the action of the centre.
	
	There is already a theorem about the uniqueness of Soergel's grading in [RS14]. Precisely, if two graded covers of $\mc{O}$ are both compatible with the action of the centre of the universal enveloping algebra, they are cover equivalent.
	
	This paper proves a similar result but for the deformed Category $\mc{O}$. This is done in two parts: existence and uniqueness. The result in [RS14] follows as a corollary to the uniqueness theorem in this paper.
	
	Let $Z$ denote the centre of the enveloping algebra, $\mc{D}$....
	A deformation ring is required in the definition of deformed $\mc{O}$. In the first instance, this will be a ring of truncated Taylor expansions. This ensures $\mc{D}$ is a finite length category [???]. The strategy is first to carefully apply commutative algebra to construct a $Z$ compatible grading on the combinatorial category of bimodules. A technical definition of a pullback is then used to show existence of a graded cover of $\mc{D}$. The proof of uniqueness is essentially elementary, however it is technical.
	
	[General Case of T = S(h)_0]
	
	Finally, there is a corollary which gives an independent proof of the result in [RS14]. The proof uses techniques developed earlier in the paper and general propositions concerning graded covers given in [RS14].
	\section{Existence of a Graded Cover}
	
	\section{Uniqueness of the Graded Cover}


\end{document}
